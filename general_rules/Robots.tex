%%%%%%%%%%%%%%%%%%%%%%%%%%%%%%%%%%%%%%%%%%%%%%%%%%%%%%%%%
\section{Robots}
\label{rule:robots}

\subsection{Number of robots}
\label{rule:robots_number}

\begin{enumerate}
	\item \textbf{Registration:} The maximum \term{number of robots} per team is \emph{two} (2).
	\item \textbf{Regular Tests:} Only one robot is allowed per test. For different tests different robots can be used.
	\item \textbf{Open Demonstrations:} In the \iterm{Finals} both robots can be used simultaneously.
\end{enumerate}

\subsection{Appearance and safety}
\label{rule:robot_appearance}

Robots should have a nice product-like appearance, be safe to operate, and should not annoy people. The following rules apply to all robots and are part of the \iterm{Robot Inspection} test (see~\refsec{sec:robot_inspection}).
\begin{enumerate}
	\item \textbf{Cover:} The robot's internal hardware (electronics and cables) should be covered in an appealing way. The use of (visible) duct tape is strictly prohibited.
	\item \textbf{Loose cables:} Loose cables hanging out of the robot are not permitted.
	\item \textbf{Safety:} The robot must not have sharp edges or elements that might harm people.
	\item \textbf{Annoyance:} The robot must not be continuously making loud noises or use blinding lights.
	\item \textbf{Marks:} The robot may not exhibit any kind of artificial marks or patterns.
	\item \textbf{Driving:} To be safe, the robots should be careful when driving (obstacle avoidance is mandatory).
\end{enumerate}

\input{general_rules/Robots-SPL}

\input{general_rules/Robots-OPL}



% Local Variables:
% TeX-master: "../Rulebook"
% End:
